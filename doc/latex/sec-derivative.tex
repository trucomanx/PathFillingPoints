\section{Derivatives}

This section presents various derivatives that are integral to the methodology of this research.

%%%%%%%%%%%%%%%%%%%%%%%%%%%%%%%%%%%%%%%%%%%%%%%%%%%%%%%%%%%%%%%%%%%%%%%%%%%%%%%%
\subsection{Derivative of $\mathbf{f}\left(\mathbf{A}\mathbf{w}-\mathbf{c}\right)$ with respect to the scalar parameter $t$}

Given 
a matrix $\mathbf{A} \in \mathbb{R}^{M\times N}$,
a vector $\mathbf{w} \in \mathbb{R}^{N}$,
a vector $\mathbf{c} \in \mathbb{R}^{M}$,
a vectorial function $\mathbf{f}:\mathbb{R}^{M} \to \mathbb{R}^{L}$ with variable vectorial
and 
the variable $t \in \mathbb{R}$. 
Then

\begin{align}
\label{eq:fAwc}
\frac{\partial \mathbf{f}\left(\mathbf{A}\mathbf{w}-\mathbf{c}\right)}{\partial t}
&=
\begin{bmatrix}
\frac{\partial f_{1}\left(\mathbf{A}\mathbf{w}-\mathbf{c}\right)}{\partial t}\\[4pt]
\frac{\partial f_{2}\left(\mathbf{A}\mathbf{w}-\mathbf{c}\right)}{\partial t}\\[4pt]
\vdots\\[4pt]
\frac{\partial f_{l}\left(\mathbf{A}\mathbf{w}-\mathbf{c}\right)}{\partial t}\\[4pt]
\vdots
\frac{\partial f_{L}\left(\mathbf{A}\mathbf{w}-\mathbf{c}\right)}{\partial t}\\[4pt]
\end{bmatrix}
\end{align}

Using the chain rule \cite[pp. 15]{petersen2008matrix} and 
given 
a function $f:\mathbb{R}^{M} \to \mathbb{R}$ 
a vector $\mathbf{v} \in \mathbb{R}^{M}$ 
and 
the variable $t \in \mathbb{R}$,
we know that

\begin{equation}
\frac{\partial f\left(\mathbf{v}\right)}{\partial t} = 
\sum_{m}^{M}
\frac{\partial f\left(\mathbf{v}\right)}{\partial v_m}
\frac{\partial v_m}{\partial t}
\end{equation}

Thus, if we define 
\begin{equation}
[\dots,~v_{m},~\dots]^{T}=\mathbf{v}=\mathbf{A}\mathbf{w}-\mathbf{c},
\end{equation}
then

\begin{align}
\frac{\partial f_{l}\left(\mathbf{A}\mathbf{w}-\mathbf{c}\right)}{\partial t}
&=
\frac{\partial f_{l}\left(\mathbf{v}\right)}{\partial t}\\[4pt]
&=
\sum_{m}^{M}
\frac{\partial f_{l}\left(\mathbf{v}\right)}{\partial v_m}
\frac{\partial v_m}{\partial t}
\end{align}

If we define $\boxed{\mathbf{1}}_{m}\equiv [\dots,~0,~1,0,\dots]^{T}$,
a vector with a unique 1 in the position $m$. 
Then $v_m = \boxed{\mathbf{1}}_{m}^{T}\left(\mathbf{A}\mathbf{w}-\mathbf{c}\right)$,
so that the last equation can be written as

\begin{align}
\frac{\partial f_{l}\left(\mathbf{A}\mathbf{w}-\mathbf{c}\right)}{\partial t}
&=
\sum_{m}^{M}
\frac{\partial f_{l}\left(\mathbf{v}\right)}{\partial v_m}
\frac{\partial \boxed{\mathbf{1}}_{m}^{T}\left(\mathbf{A}\mathbf{w}-\mathbf{c}\right)}{\partial t}\\[4pt]
&=
\sum_{m}^{M}
\frac{\partial f_{l}\left(\mathbf{v}\right)}{\partial v_m}
\boxed{\mathbf{1}}_{m}^{T}\left(\mathbf{A}\frac{\partial \mathbf{w}}{\partial t}\right)\\[4pt]
&=
\left\{
\sum_{m}^{M}
\frac{\partial f_{l}\left(\mathbf{v}\right)}{\partial v_m}
\boxed{\mathbf{1}}_{m}^{T}
\right\}
\mathbf{A}\frac{\partial \mathbf{w}}{\partial t}\\[4pt]
&=
\begin{bmatrix}
\frac{\partial f_{l}\left(\mathbf{v}\right)}{\partial v_1} &
%\frac{\partial f_{l}\left(\mathbf{v}\right)}{\partial v_2} &
\dots &
\frac{\partial f_{l}\left(\mathbf{v}\right)}{\partial v_m} &
\dots &
\frac{\partial f_{l}\left(\mathbf{v}\right)}{\partial v_M} 
\end{bmatrix}^{T}
\mathbf{A}\frac{\partial \mathbf{w}}{\partial t}\\[4pt]
&=
\nabla^{T} f_{l}\left(\mathbf{v}\right)
\mathbf{A}\frac{\partial \mathbf{w}}{\partial t}
\end{align}

Appplying in the Eq. \ref{eq:fAwc}, we obtain

\begin{align}
\frac{\partial \mathbf{f}\left(\mathbf{A}\mathbf{w}-\mathbf{c}\right)}{\partial t}
&=
\begin{bmatrix}
\nabla^{T} f_{1}\left(\mathbf{v}\right)
\mathbf{A}\frac{\partial \mathbf{w}}{\partial t}\\[4pt]
\nabla^{T} f_{2}\left(\mathbf{v}\right)
\mathbf{A}\frac{\partial \mathbf{w}}{\partial t}\\[4pt]
\vdots\\[4pt]
\nabla^{T} f_{l}\left(\mathbf{v}\right)
\mathbf{A}\frac{\partial \mathbf{w}}{\partial t}\\[4pt]
\vdots
\nabla^{T} f_{L}\left(\mathbf{v}\right)
\mathbf{A}\frac{\partial \mathbf{w}}{\partial t}\\[4pt]
\end{bmatrix}\\
&=
\begin{bmatrix}
\nabla^{T} f_{1}\left(\mathbf{v}\right)\\[4pt]
\nabla^{T} f_{2}\left(\mathbf{v}\right)\\[4pt]
\vdots\\[4pt]
\nabla^{T} f_{l}\left(\mathbf{v}\right)\\[4pt]
\vdots
\nabla^{T} f_{L}\left(\mathbf{v}\right)\\[4pt]
\end{bmatrix}
\mathbf{A}\frac{\partial \mathbf{w}}{\partial t}\\
\end{align}

\begin{equation}
\label{eq:DfAwc}
\frac{\partial \mathbf{f}\left(\mathbf{A}\mathbf{w}-\mathbf{c}\right)}{\partial t}
=
\nabla \mathbf{f}(\mathbf{v})
\mathbf{A}\frac{\partial \mathbf{w}}{\partial t}
\end{equation}

where $\nabla \mathbf{f}(\mathbf{v})$ is the Jacobian matrix of $\mathbf{f}(\mathbf{v})$. 

%%%%%%%%%%%%%%%%%%%%%%%%%%%%%%%%%%%%%%%%%%%%%%%%%%%%%%%%%%%%%%%%%%%%%%%%%%%%%%%%
\subsection{Derivative of $\left\|\mathbf{f}\left(\mathbf{A}\mathbf{w}-\mathbf{c}\right)\right\|_{\mathbf{Q}}^2$ with respect to the scalar parameter $t$}

Given a diagonal matrix $\mathbf{Q} \in \mathbb{R}^{L\times L}$,
a matrix $\mathbf{A} \in \mathbb{R}^{M\times N}$,
a vector $\mathbf{w} \in \mathbb{R}^{N}$,
a vector $\mathbf{c} \in \mathbb{R}^{M}$ and 
a vectorial function $\mathbf{f}:\mathbb{R}^{M} \to \mathbb{R}^{L}$.

\begin{align}
E&=\left\|\mathbf{f}\left(\mathbf{A}\mathbf{w}-\mathbf{c}\right)\right\|_{\mathbf{Q}}^2\\
~&=\mathbf{f}^{T}\left(\mathbf{A}\mathbf{w}-\mathbf{c}\right)\mathbf{Q}\mathbf{f}\left(\mathbf{A}\mathbf{w}-\mathbf{c}\right)
\end{align}

If we define $\mathbf{A}$, $\mathbf{Q}$, $\mathbf{c}$, and $\mathbf{f}$ not to depend on the variable $t$.

\begin{align}
\frac{\partial E}{\partial t}
&=\frac{\partial\left\{ \mathbf{f}^{T}\left(\mathbf{A}\mathbf{w}-\mathbf{c}\right)\mathbf{Q}\mathbf{f}\left(\mathbf{A}\mathbf{w}-\mathbf{c}\right)\right\}}{\partial t} \\
~
&=
\left\{\frac{\partial \mathbf{f}\left(\mathbf{A}\mathbf{w}-\mathbf{c}\right)}{\partial t} \right\}^{T}
\mathbf{Q}\mathbf{f}\left(\mathbf{A}\mathbf{w}-\mathbf{c}\right)
+
\mathbf{f}^{T}\left(\mathbf{A}\mathbf{w}-\mathbf{c}\right)\mathbf{Q}
\frac{\partial\left\{ \mathbf{f}\left(\mathbf{A}\mathbf{w}-\mathbf{c}\right)\right\}}{\partial t} \\
~
&=
2
\mathbf{f}^{T}\left(\mathbf{A}\mathbf{w}-\mathbf{c}\right)\mathbf{Q}
\frac{\partial \mathbf{f}\left(\mathbf{A}\mathbf{w}-\mathbf{c}\right)}{\partial t} 
\end{align}

Applying the Eq. \ref{eq:DfAwc} with $\mathbf{v}=\mathbf{A}\mathbf{w}-\mathbf{c}$

\begin{align}
\label{eq:dEdt}
\frac{\partial E}{\partial t}
&=
2
\mathbf{f}^{T}\left(\mathbf{A}\mathbf{w}-\mathbf{c}\right)\mathbf{Q}
\nabla \mathbf{f}(\mathbf{v})
\mathbf{A}\frac{\partial \mathbf{w}}{\partial t}\\
&=
2
\mathbf{f}^{T}\left(\mathbf{A}\mathbf{w}-\mathbf{c}\right)\mathbf{Q}
\mathbf{J}(\mathbf{v})
\mathbf{A}\frac{\partial \mathbf{w}}{\partial t}
\end{align}

Where $\mathbf{J}(\mathbf{v}) \equiv \nabla \mathbf{f}(\mathbf{v})$ is the Jacobian matrix of the function $\mathbf{f}(\mathbf{v})$.

%%%%%%%%%%%%%%%%%%%%%%%%%%%%%%%%%%%%%%%%%%%%%%%%%%%%%%%%%%%%%%%%%%%%%%%%%%%%%%%%
\subsection{Derivative of $\left\|\mathbf{f}\left(\mathbf{A}\mathbf{w}-\mathbf{c}\right)\right\|_{\mathbf{Q}}^2$
with respect to the vector $\mathbf{w}$}
\label{subsec:funcAwcQ2w}

Given the expression $E=\left\|\mathbf{f}\left(\mathbf{A}\mathbf{w}-\mathbf{c}\right)\right\|_{\mathbf{Q}}^2$,
where
the matrix $\mathbf{A} \in \mathbb{R}^{M\times N}$,
the vector $\mathbf{w} \in \mathbb{R}^{N}$,
the vector $\mathbf{c} \in \mathbb{R}^{M}$ and 
the vectorial function $\mathbf{f}:\mathbb{R}^{M} \to \mathbb{R}^{L}$.
If we define $\mathbf{A}$, $\mathbf{Q}$, $\mathbf{c}$ and $\mathbf{f}$ not depend on vector $\mathbf{w}$. 
Then 

\begin{align}
\frac{\partial E}{\partial \mathbf{w}}
&=
\begin{bmatrix}
\frac{\partial E}{\partial w_{1}}&
\frac{\partial E}{\partial w_{1}}&
\dots&
\frac{\partial E}{\partial w_{n}}&
\dots&
\frac{\partial E}{\partial w_{N}}
\end{bmatrix}^{T}
\end{align}

Using the Eq. \ref{eq:dEdt}
\begin{align}
\frac{\partial E}{\partial \mathbf{w}}
&=
\left\{
2
\mathbf{f}^{T}\left(\mathbf{A}\mathbf{w}-\mathbf{c}\right)\mathbf{Q}
\nabla \mathbf{f}(\mathbf{v})
\mathbf{A}
\begin{bmatrix}
\frac{\partial \mathbf{w}}{\partial w_{1}}&
\frac{\partial \mathbf{w}}{\partial w_{2}}&
\dots&
\frac{\partial \mathbf{w}}{\partial w_{n}}&
\dots&
\frac{\partial \mathbf{w}}{\partial w_{N}}
\end{bmatrix}
\right\}^{T}\\[4pt]
&=
\left\{
2
\mathbf{f}^{T}\left(\mathbf{A}\mathbf{w}-\mathbf{c}\right)\mathbf{Q}
\nabla \mathbf{f}(\mathbf{v})
\mathbf{A}
\right\}^{T}\\[4pt]
&=
2
\left\{
\mathbf{Q}
\nabla \mathbf{f}(\mathbf{v})
\mathbf{A}
\right\}^{T}
\mathbf{f}\left(\mathbf{A}\mathbf{w}-\mathbf{c}\right)\\[4pt]
&=
2
\mathbf{A}^{T}
\left\{
\mathbf{Q}
\nabla \mathbf{f}(\mathbf{v})
\right\}^{T}
\mathbf{f}\left(\mathbf{A}\mathbf{w}-\mathbf{c}\right)
\end{align}

\begin{align}
\frac{\partial E}{\partial \mathbf{w}}
&=
2
\mathbf{A}^{T}
\left\{
\nabla \mathbf{f}(\mathbf{v})
\right\}^{T}
\mathbf{Q}
\mathbf{f}\left(\mathbf{A}\mathbf{w}-\mathbf{c}\right)\\
&=
2
\mathbf{A}^{T}
\mathbf{J}\left(\mathbf{A}\mathbf{w}-\mathbf{c}\right)^{T}
\mathbf{Q}
\mathbf{f}\left(\mathbf{A}\mathbf{w}-\mathbf{c}\right)
\end{align}

Where $\mathbf{J}\left(\mathbf{A}\mathbf{w}-\mathbf{c}\right)=\nabla \mathbf{f}(\mathbf{v})$ is the Jacobian matrix of the vectorial function $\mathbf{f}\left(\mathbf{A}\mathbf{w}-\mathbf{c}\right)$.

%%%%%%%%%%%%%%%%%%%%%%%%%%%%%%%%%%%%%%%%%%%%%%%%%%%%%%%%%%%%%%%%%%%%%%%%%%%%%%%%
\subsection{Derivative for a specific type of vector function $\mathbf{f}(\mathbf{v})$ with respect to the vector $\mathbf{v}$}

Given 
a vector $\mathbf{v} \in \mathbb{R}^{M}$ and 
a vectorial function $\mathbf{f}:\mathbb{R}^{M} \to \mathbb{R}^{M}$ with vector parameter 
$\mathbf{v}\equiv \left[v_{1}, v_{2}, \dots,v_{m},\dots v_{M}\right]^{T}$.
If we define 
\begin{equation}
\mathbf{f}(\mathbf{v}) 
\equiv
\begin{bmatrix}
f_{1}(v_{1}) &
f_{2}(v_{2}) &
\dots&
f_{m}(v_{m}) &
\dots&
f_{M}(v_{M}) &
\end{bmatrix}^{T},
\end{equation}
where $f_{m}:\mathbb{R} \to \mathbb{R}$, $\forall 1 \leq m\leq M$.
Then, we can calculate the Jacobian matrix of function $\mathbf{f}(\mathbf{v})$ as

\begin{equation}
\nabla\mathbf{f}(\mathbf{v}) 
\equiv
\begin{bmatrix}
\frac{\partial f_{1}(v_{1})}{\partial v_{1}} & 0  & \dots & 0      & 0\\
0  & \frac{\partial f_{2}(v_{2})}{\partial v_{2}} & \dots & 0      & 0\\
\vdots                  & \vdots                  & \dots & \vdots & \vdots\\
0                       & 0                       & \dots & \frac{\partial f_{M-1}(v_{M-1})}{\partial v_{M-1}} & 0\\
0                       & 0                       & \dots & 0      & \frac{\partial f_{M}(v_{M})}{\partial v_{M}}
\end{bmatrix},
\end{equation}

\begin{equation}
\left\{\nabla \mathbf{f}(\mathbf{v}) \right\}^{T}
\equiv
\nabla \mathbf{f}(\mathbf{v}) 
\equiv
diag
\left(
\begin{bmatrix}
\frac{\partial f_{1}(v_{1})}{\partial v_{1}} &
\frac{\partial f_{2}(v_{2})}{\partial v_{2}} &
\dots&
\frac{\partial f_{m}(v_{m})}{\partial v_{m}} &
\dots&
\frac{\partial f_{M}(v_{M})}{\partial v_{M}} &
\end{bmatrix}
\right),
\end{equation}

%%%%%%%%%%%%%%%%%%%%%%%%%%%%%%%%%%%%%%%%%%%%%%%%%%%%%%%%%%%%%%%%%%%%%%%%%%%%%%%%
\subsection{Derivativer of square curvature $\mathcal{K}^{2}(\mathbf{\bar{w}})$ with respect to the vector $\mathbf{\bar{w}}$}
If we define
\begin{equation}
\mathcal{K}^{2}(\mathbf{\bar{w}})
\equiv
\frac{\left\| \mathbf{\bar{b}} \times \mathbf{\bar{c}} \right\|^{3}}{\left\| \mathbf{\bar{b}} \right\|^{6}},
\end{equation}
where $\mathbf{\bar{b}} \equiv \mathbf{Q}^{(1)} \mathbf{\bar{w}}$ and $\mathbf{\bar{c}} \equiv \mathbf{Q}^{(2)} \mathbf{\bar{w}}$. The matrices $\mathbf{Q}^{(1)}$ and $\mathbf{Q}^{(1)}$ are constants. 
Using the Eq. (\ref{eq:propcross}), we obtain
\begin{equation}
\mathcal{K}^{2}(\mathbf{\bar{w}})
\equiv
\frac{
\left\|
\mathbf{\bar{b}} 
\right\|^{2}
\left\|
\mathbf{\bar{c}}
\right\|^{2}
-
\left(
\mathbf{\bar{b}}^{T}
\mathbf{\bar{c}}
\right)^{2}
}
{\left\| \mathbf{\bar{b}} \right\|^{6}},
\end{equation}

Derivating $\mathcal{K}^{2}(\mathbf{\bar{w}})$ in relation to vector $\mathbf{\bar{w}}$

\begin{equation}
\frac{
\partial 
\mathcal{K}^{2}(\mathbf{\bar{w}})
}
{
\partial \mathbf{\bar{w}}
}
=
\frac{
\frac{
\partial 
\left(
\left\|
\mathbf{\bar{b}} 
\right\|^{2}
\left\|
\mathbf{\bar{c}}
\right\|^{2}
-
\left(
\mathbf{\bar{b}}^{T}
\mathbf{\bar{c}}
\right)^{2}
\right)
}
{\partial \mathbf{\bar{w}}}
\left\| \mathbf{\bar{b}} \right\|^{6}
-
\left(
\left\|
\mathbf{\bar{b}} 
\right\|^{2}
\left\|
\mathbf{\bar{c}}
\right\|^{2}
-
\left(
\mathbf{\bar{b}}^{T}
\mathbf{\bar{c}}
\right)^{2}
\right)
\frac{
\partial
\left\| \mathbf{\bar{b}} \right\|^{6}
}
{
\partial \mathbf{\bar{w}}
}
}
{\left\| \mathbf{\bar{b}} \right\|^{12}}
\end{equation}



\begin{equation}\label{eq:curve2}
\frac{
\partial 
\mathcal{K}^{2}(\mathbf{\bar{w}})
}
{
\partial \mathbf{\bar{w}}
}
=
\frac{
\left(
\frac{
\partial 
\left(
\left\|\mathbf{\bar{b}}\right\|^2
\left\|\mathbf{\bar{c}}\right\|^2
\right)
}
{\partial \mathbf{\bar{w}}}
-
\frac{
\partial
\left(
\mathbf{\bar{b}}^{T}
\mathbf{\bar{c}}
\right)^{2}
}
{\partial \mathbf{\bar{w}}}
\right)
\left\| \mathbf{\bar{b}} \right\|^{6}
-
\left(
\left\|
\mathbf{\bar{b}} 
\right\|^{2}
\left\|
\mathbf{\bar{c}}
\right\|^{2}
-
\left(
\mathbf{\bar{b}}^{T}
\mathbf{\bar{c}}
\right)^{2}
\right)
\frac{
\partial
\left\| \mathbf{\bar{b}} \right\|^{6}
}
{
\partial \mathbf{\bar{w}}
}
}
{\left\| \mathbf{\bar{b}} \right\|^{12}}
\end{equation}

Solving \cite[pp. 11]{petersen2008matrix} the Eq. \ref{eq:curve2} by parts

\begin{align}
\frac{
\partial 
\left(
\left\|\mathbf{\bar{b}}\right\|^2
\left\|\mathbf{\bar{c}}\right\|^2
\right)
}
{\partial \mathbf{\bar{w}}}
&=
\frac{
\partial 
\left\|\mathbf{\bar{b}}\right\|^2
}
{\partial \mathbf{\bar{w}}}
\left\|\mathbf{\bar{c}}\right\|^2
+
\left\|\mathbf{\bar{b}}\right\|^2
\frac{
\partial 
\left\|\mathbf{\bar{c}}\right\|^2
}
{\partial \mathbf{\bar{w}}}\\
~
&=
\frac{
\partial 
\left(
\mathbf{\bar{w}}^{T} \mathbf{Q}^{(1)T}
\mathbf{Q}^{(1)} \mathbf{\bar{w}}
\right)
}
{\partial \mathbf{\bar{w}}}
\left\|\mathbf{\bar{c}}\right\|^2
+
\left\|\mathbf{\bar{b}}\right\|^2
\frac{
\partial 
\left(
\mathbf{\bar{w}}^{T} \mathbf{Q}^{(2)T}
\mathbf{Q}^{(2)} \mathbf{\bar{w}}
\right)
}
{\partial \mathbf{\bar{w}}}\\
~
&=
2
\mathbf{Q}^{(1)T} \mathbf{Q}^{(1)} \mathbf{\bar{w}}
\left\|\mathbf{\bar{c}}\right\|^2
+
2
\left\|\mathbf{\bar{b}}\right\|^2
\mathbf{Q}^{(2)T} \mathbf{Q}^{(2)} \mathbf{\bar{w}}\\
~
&=
2
\left\|\mathbf{\bar{c}}\right\|^2
\mathbf{Q}^{(1)T} \mathbf{\bar{b}}
+
2
\left\|\mathbf{\bar{b}}\right\|^2
\mathbf{Q}^{(2)T} \mathbf{\bar{c}}
\end{align}

\begin{align}
\frac{
\partial
\left(
\mathbf{\bar{b}}^{T}
\mathbf{\bar{c}}
\right)^{2}
}
{\partial \mathbf{\bar{w}}} 
&=
2
\left(
\mathbf{\bar{b}}^{T}
\mathbf{\bar{c}}
\right)
\frac{
\partial
\left(
\mathbf{\bar{w}}^{T} \mathbf{Q}^{(1)T}
\mathbf{Q}^{(2)} \mathbf{\bar{w}}
\right)
}
{\partial \mathbf{\bar{w}}}\\
~
&=
2
\left(
\mathbf{\bar{b}}^{T}
\mathbf{\bar{c}}
\right)
\left(
\mathbf{Q}^{(1)T}\mathbf{Q}^{(2)}
+
\mathbf{Q}^{(2)T}\mathbf{Q}^{(1)}
\right)
\mathbf{\bar{w}}\\
~
&=
2
\left(
\mathbf{\bar{b}}^{T}
\mathbf{\bar{c}}
\right)
\left(
\mathbf{Q}^{(1)T}\mathbf{\bar{c}}
+
\mathbf{Q}^{(2)T}\mathbf{\bar{b}}
\right)
\end{align}


\begin{align}
\frac{
\partial
\left\| \mathbf{\bar{b}} \right\|^{6}
}
{
\partial \mathbf{\bar{w}}
}
&=
\frac{
\partial
\left( \left\| \mathbf{\bar{b}} \right\|^{2} \right)^{3}
}
{
\partial \mathbf{\bar{w}}
}\\
~
&=
3
\left( \left\| \mathbf{\bar{b}} \right\|^{2} \right)^{2}
\frac{
\partial
\left\| \mathbf{\bar{b}} \right\|^{2}
}
{
\partial \mathbf{\bar{w}}
}\\
~
&=
3
\left\| \mathbf{\bar{b}} \right\|^{4}
\frac{
\partial
\mathbf{\bar{b}}^{T}\mathbf{\bar{b}}
}
{
\partial \mathbf{\bar{w}}
}\\
~
&=
3
\left\| \mathbf{\bar{b}} \right\|^{4}
\frac{
\partial
\mathbf{\bar{w}}^{T}\mathbf{Q}^{(1)T}\mathbf{Q}^{(1)}\mathbf{\bar{w}}
}
{
\partial \mathbf{\bar{w}}
}\\
~
&=
6
\left\| \mathbf{\bar{b}} \right\|^{4}
\mathbf{Q}^{(1)T}\mathbf{Q}^{(1)}\mathbf{\bar{w}}\\
~
&=
6
\left\| \mathbf{\bar{b}} \right\|^{4}
\mathbf{Q}^{(1)T}\mathbf{\bar{b}}
\end{align}

We obtain

\tiny
\begin{equation}
\frac{
\partial 
\mathcal{K}^{2}(\mathbf{\bar{w}})
}
{
\partial \mathbf{\bar{w}}
}
=
\frac{
\left[
2
\left\|\mathbf{\bar{c}}\right\|^2
\mathbf{Q}^{(1)T} \mathbf{\bar{b}}
+
2
\left\|\mathbf{\bar{b}}\right\|^2
\mathbf{Q}^{(2)T} \mathbf{\bar{c}}
-
2
\left(
\mathbf{\bar{b}}^{T}
\mathbf{\bar{c}}
\right)
\left(
\mathbf{Q}^{(1)T}\mathbf{\bar{c}}
+
\mathbf{Q}^{(2)T}\mathbf{\bar{b}}
\right)
\right]
\left\| \mathbf{\bar{b}} \right\|^{6}
-
\left(
\left\|
\mathbf{\bar{b}} 
\right\|^{2}
\left\|
\mathbf{\bar{c}}
\right\|^{2}
-
\left(
\mathbf{\bar{b}}^{T}
\mathbf{\bar{c}}
\right)^{2}
\right)
6
\left\| \mathbf{\bar{b}} \right\|^{4}
\mathbf{Q}^{(1)T}\mathbf{\bar{b}}
}
{\left\| \mathbf{\bar{b}} \right\|^{12}}
\end{equation}
\normalsize

\small
\begin{equation}
\frac{
\partial 
\mathcal{K}^{2}(\mathbf{\bar{w}})
}
{
\partial \mathbf{\bar{w}}
}
=
\frac{
\left[
2
\left\|\mathbf{\bar{c}}\right\|^2
\mathbf{Q}^{(1)T} \mathbf{\bar{b}}
+
2
\left\|\mathbf{\bar{b}}\right\|^2
\mathbf{Q}^{(2)T} \mathbf{\bar{c}}
-
2
\left(
\mathbf{\bar{b}}^{T}
\mathbf{\bar{c}}
\right)
\left(
\mathbf{Q}^{(1)T}\mathbf{\bar{c}}
+
\mathbf{Q}^{(2)T}\mathbf{\bar{b}}
\right)
\right]
\left\| \mathbf{\bar{b}} \right\|^{6}
-
\mathcal{K}^{2}(\mathbf{\bar{w}})
6
\left\| \mathbf{\bar{b}} \right\|^{10}
\mathbf{Q}^{(1)T}\mathbf{\bar{b}}
}
{\left\| \mathbf{\bar{b}} \right\|^{12}}
\end{equation}
\normalsize



Finally
\small
\begin{equation}\label{eq:cuvaturepartial}
\frac{
\partial 
\mathcal{K}^{2}(\mathbf{\bar{w}})
}
{
\partial \mathbf{\bar{w}}
}
=
\frac{
2
\left\|\mathbf{\bar{c}}\right\|^2
\mathbf{Q}^{(1)T} \mathbf{\bar{b}}
+
2
\left\|\mathbf{\bar{b}}\right\|^2
\mathbf{Q}^{(2)T} \mathbf{\bar{c}}
}
{\left\| \mathbf{\bar{b}} \right\|^{6}}
-
\frac
{
2
\left(
\mathbf{\bar{b}}^{T}
\mathbf{\bar{c}}
\right)
\left(
\mathbf{Q}^{(1)T}\mathbf{\bar{c}}
+
\mathbf{Q}^{(2)T}\mathbf{\bar{b}}
\right)
}
{\left\| \mathbf{\bar{b}} \right\|^{6}}
-
\frac
{
6
\mathcal{K}^{2}(\mathbf{\bar{w}})
\mathbf{Q}^{(1)T}\mathbf{\bar{b}}
}
{\left\| \mathbf{\bar{b}} \right\|^{2}}
\end{equation}
\normalsize