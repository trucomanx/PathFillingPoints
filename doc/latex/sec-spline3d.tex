
%%%%%%%%%%%%%%%%%%%%%%%%%%%%%%%%%%%%%%%%%%%%%%%%%%%%%%%%%%%%%%%%%%%%%%%%%%%%%
%%%%%%%%%%%%%%%%%%%%%%%%%%%%%%%%%%%%%%%%%%%%%%%%%%%%%%%%%%%%%%%%%%%%%%%%%%%%%
\section{3D parametric cubic spline }
A 3D parametric cubic spline $\mathbf{\bar{p}}: [0,N-1] \subseteq \mathbb{R} \to \mathbb{R}^3$ is a parametric function defined piecewise by parametric polynomials in $\mathbb{R}^{3}$ space where $t_{n}\equiv n$,
see Section \ref{sec:curvePiecewise}.
The Fig. \ref{fig:3DSplinePoly} shows the polynomials $\mathbf{p}^{(n)}(t)$ with parameter $t$ 
close to the n-th position.
\begin{figure}[H]
    \centering
    \includegraphics[width=0.35\textwidth]{boveda/Diagrama1.eps}
    \caption{Polynomials in the n-th position of cubic spline}
    \label{fig:3DSplinePoly}
\end{figure}

Given a set of $N$ points $\mathbf{r}^{(n)}\in\mathbb{R}^{3}$, $\forall$ $0\leq n \leq N-1$, 
we can generate a cubic spline with $N-2$ polynomials $\mathbf{p}^{(n)}(t)$, $\forall$ $0\leq n \leq N-2$, 
according to Fig. \ref{fig:3DSplinePoly}.
So that, it is fulfilled that 

\begin{equation}
\mathbf{p}^{(n)}(t)=
\begin{bmatrix}
x^{(n)}(t) & y^{(n)}(t) & z^{(n)}(t)
\end{bmatrix}^{T},
\end{equation}
where
\begin{equation}
x^{(n)}(t)=a_{x}^{(n)}+b_{x}^{(n)}(t-n)+c_{x}^{(n)}(t-n)^{2}+d_{x}^{(n)}(t-n)^{3},
\end{equation}
\begin{equation}
y^{(n)}(t)=a_{y}^{(n)}+b_{y}^{(n)}(t-n)+c_{y}^{(n)}(t-n)^{2}+d_{y}^{(n)}(t-n)^{3},
\end{equation}
\begin{equation}
z^{(n)}(t)=a_{z}^{(n)}+b_{z}^{(n)}(t-n)+c_{z}^{(n)}(t-n)^{2}+d_{z}^{(n)}(t-n)^{3}.
\end{equation}

%%%%%%%%%%%%%%%%%%%%%%%%%%%%%%%%%%%%%%%%%%%%%%%%%%%%%%%%%%%%%%%%%%%%%%%%%%%%%
\subsection{Matricial form of polynomials $\mathbf{p}^{(n)}(t)$}
For all values $0\leq n \leq N-2$, we know that
\begin{equation}
\mathbf{w}^{(n)}=
\left[
\begin{array}{cccc:cccc:cccc}
a_{x}^{(n)} & b_{x}^{(n)} & c_{x}^{(n)} & d_{x}^{(n)} & 
a_{y}^{(n)} & b_{y}^{(n)} & c_{y}^{(n)} & d_{y}^{(n)} & 
a_{z}^{(n)} & b_{z}^{(n)} & c_{z}^{(n)} & d_{z}^{(n)}
\end{array}
\right]^{T},
\end{equation}
\small
\begin{equation}\label{eq:Ant}
\mathbf{A}^{(n)}(t)
=
\left[
\begin{array}{cccc:cccc:cccc}
1 & (t-n) & (t-n)^{2} & (t-n)^{3} &
0 & 0 & 0 & 0 &
0 & 0 & 0 & 0 \\
0 & 0 & 0 & 0 &
1 & (t-n) & (t-n)^{2} & (t-n)^{3} &
0 & 0 & 0 & 0 \\
0 & 0 & 0 & 0 &
0 & 0 & 0 & 0 &
1 & (t-n) & (t-n)^{2} & (t-n)^{3} 
\end{array}
\right],
\end{equation}
\normalsize

\begin{equation}\label{eq:primeder0}
\mathbf{p}^{(n)}(t)=
\mathbf{A}^{(n)}(t) \mathbf{w}^{(n)},
\end{equation}

Where, $\mathbf{A}^{(n)}(t) \in \mathbb{R}^{3 \times 12}$ and $\mathbf{w}^{(n)} \in \mathbb{R}^{12}$.
Additionally, we can define

\begin{equation}\label{eq:wvector}
\mathbf{w}
\equiv
\begin{bmatrix}
\mathbf{w}^{(0)}\\
\mathbf{w}^{(1)}\\
%\mathbf{w}^{(2)}\\
\vdots\\
\mathbf{w}^{(N-2)}\\
\end{bmatrix}
\in \mathbb{R}^{12(N-1)}
\end{equation}

%%%%%%%%%%%%%%%%%%%%%%%%%%%%%%%%%%%%%%%%%%%%%%%%%%%%%%%%%%%%%%%%%%%%%%%%%%%%%
\subsection{Derivative of $\mathbf{p}^{(n)}(t)$ with respect to escalar $t$}

\small
\begin{equation}\label{eq:primeder1}
\frac{\partial \mathbf{A}^{(n)}(t)}{\partial t}=
\left[
\begin{array}{cccc:cccc:cccc}
0 & 1 & 2(t-n) & 3(t-n)^{2} &
0 & 0 & 0 & 0 &
0 & 0 & 0 & 0 \\
0 & 0 & 0 & 0 &
0 & 1 & 2(t-n) & 3(t-n)^{2} &
0 & 0 & 0 & 0 \\
0 & 0 & 0 & 0 &
0 & 0 & 0 & 0 &
0 & 1 & 2(t-n) & 3(t-n)^{2} 
\end{array}
\right],
\end{equation}
\normalsize

\small
\begin{equation}\label{eq:primeder2}
\frac{\partial^{2} \mathbf{A}^{(n)}(t)}{\partial t^{2}}=
\left[
\begin{array}{cccc:cccc:cccc}
0 & 0 & 2 & 6(t-n) &
0 & 0 & 0 & 0 &
0 & 0 & 0 & 0 \\
0 & 0 & 0 & 0 &
0 & 0 & 2 & 6(t-n) &
0 & 0 & 0 & 0 \\
0 & 0 & 0 & 0 &
0 & 0 & 0 & 0 &
0 & 0 & 2 & 6(t-n) 
\end{array}
\right].
\end{equation}
\normalsize

 If we define 
 $\frac{\partial \mathbf{A}^{(n)}(t)}{\partial t} \equiv D_{t}\mathbf{A}^{(n)}(t)$
 and
 $\frac{\partial^{2} \mathbf{A}^{(n)}(t)}{\partial t^{2}} \equiv D_{t}^{2}\mathbf{A}^{(n)}(t)$, 
 then
 
\begin{align}\label{eq:primeder3a}
\frac{\partial \mathbf{p}^{(n)}(t)}{\partial t}
&=
D_{t}\mathbf{A}^{(n)}(t) \mathbf{w}^{(n)} \\
~
&=
\begin{bmatrix}
b_{x}^{(n)}+2(t-n) c_{x}^{(n)}+3(t-n)^{2} d_{x}^{(n)}\\
b_{y}^{(n)}+2(t-n) c_{y}^{(n)}+3(t-n)^{2} d_{y}^{(n)}\\
b_{z}^{(n)}+2(t-n) c_{z}^{(n)}+3(t-n)^{2} d_{z}^{(n)}
\end{bmatrix}
\end{align}


\begin{align}\label{eq:primeder3b}
\frac{\partial^{2} \mathbf{p}^{(n)}(t)}{\partial t^{2}}
&=
D_{t}^{2}\mathbf{A}^{(n)}(t) \mathbf{w}^{(n)}\\
~ 
&=
\begin{bmatrix}
2 c_{x}^{(n)}+6(t-n) d_{x}^{(n)}\\
2 c_{y}^{(n)}+6(t-n) d_{y}^{(n)}\\
2 c_{z}^{(n)}+6(t-n) d_{z}^{(n)}
\end{bmatrix}
\end{align}



%%%%%%%%%%%%%%%%%%%%%%%%%%%%%%%%%%%%%%%%%%%%%%%%%%%%%%%%%%%%%%%%%%%%%%%%%%%%%
%%%%%%%%%%%%%%%%%%%%%%%%%%%%%%%%%%%%%%%%%%%%%%%%%%%%%%%%%%%%%%%%%%%%%%%%%%%%%
\subsection{Conditions in 3D parametric cubic spline}\label{sec:boundarycubic}

%%%%%%%%%%%%%%%%%%%%%%%%%%%%%%%%%%%%%%%%%%%%%%%%%%%%%%%%%%%%%%%%%%%%%%%%%%%%%
\subsubsection{Boundary conditions in points}

Following the Fig. \ref{fig:3DSplinePoly}, 
we can affirm that for $0 \leq n\leq N-2$,
so that
\begin{equation}\label{eq:condition1}
\mathbf{p}^{(n)}(n)\approx \mathbf{r}^{(n)}
\in \mathbb{R}^{3},
\end{equation}

\begin{equation}\label{eq:condition2}
\mathbf{p}^{(N-2)}(N-1)\approx \mathbf{r}^{(N-1)}
\in \mathbb{R}^{3},
\end{equation}

\textcolor{red}{Referring to Eq. (\ref{eq:condition2}), 
we specifically focused on analyzing the initial boundary of each polynomial. 
This approach was adopted because the continuity between consecutive polynomials is thoroughly examined in the boundary continuity section, rendering further analysis redundant.}

\subsubsection{Matricial form of the boundary conditions in points}
Using 
the Eq. \ref{eq:primeder0} in 
the Eqs. \ref{eq:condition1} and \ref{eq:condition2},
we obtain for $0 \leq n\leq N-2$

\begin{equation}\label{eq:pointcond1}
\mathbf{A}^{(n)}(n) \mathbf{w}^{(n)}\approx \mathbf{r}^{(n)}
\in \mathbb{R}^{3},
\end{equation}

\begin{equation}\label{eq:pointcond2}
\mathbf{A}^{(N-2)}(N-1) \mathbf{w}^{(N-2)}\approx \mathbf{r}^{(N-1)}
\in \mathbb{R}^{3},
\end{equation}





where, using the Eq. \ref{eq:Ant}, we know that

\begin{equation}\label{eq:Q00}
\mathbf{A}^{(n)}(n+1)=
\left[
\begin{array}{cccc:cccc:cccc}
1 & 1 & 1 & 1 &
0 & 0 & 0 & 0 &
0 & 0 & 0 & 0 \\
0 & 0 & 0 & 0 &
1 & 1 & 1 & 1 &
0 & 0 & 0 & 0 \\
0 & 0 & 0 & 0 &
0 & 0 & 0 & 0 &
1 & 1 & 1 & 1 
\end{array}
\right]
\equiv \mathbf{Q}^{(0,0)}\in \mathbb{R}^{3\times 12},
\end{equation}

\begin{equation}\label{eq:Q01}
\mathbf{A}^{(n+1)}(n+1)
=
\mathbf{A}^{(n)}(n)
=
\left[
\begin{array}{cccc:cccc:cccc}
1 & 0 & 0 & 0 &
0 & 0 & 0 & 0 &
0 & 0 & 0 & 0 \\
0 & 0 & 0 & 0 &
1 & 0 & 0 & 0 &
0 & 0 & 0 & 0 \\
0 & 0 & 0 & 0 &
0 & 0 & 0 & 0 &
1 & 0 & 0 & 0 
\end{array}
\right]
\equiv \mathbf{Q}^{(0,1)}\in \mathbb{R}^{3\times 12},
\end{equation}

Thus,
using the Eqs. \ref{eq:Q00} and \ref{eq:Q01} in 
the Eqs. \ref{eq:pointcond1} and \ref{eq:pointcond2}, 
$\forall 0 \leq n\leq N-2$;
We can write the Eqs. \ref{eq:condition1} and \ref{eq:condition2} as 

\begin{equation}
\mathbf{P}
\mathbf{w}
\approx \mathbf{r}\in \mathbb{R}^{3N}.
\end{equation}

Where
\begin{equation}\label{eq:Pmat}
\mathbf{P}
\equiv
\begin{bmatrix}
\mathbf{Q}^{(0,1)} & \mathbf{0}         & \hdots & \mathbf{0} & \mathbf{0}         & \mathbf{0}\\
\mathbf{0}         & \mathbf{Q}^{(0,1)} & \hdots & \mathbf{0} & \mathbf{0}         & \mathbf{0}\\
\vdots             & \vdots             & \vdots & \vdots     & \vdots             & \vdots    \\ 
\mathbf{0}         & \mathbf{0}         & \hdots & \mathbf{0} & \mathbf{Q}^{(0,1)} & \mathbf{0}\\
\mathbf{0}         & \mathbf{0}         & \hdots & \mathbf{0} & \mathbf{0}         & \mathbf{Q}^{(0,1)}\\
\mathbf{0}         & \mathbf{0}         & \hdots & \mathbf{0} & \mathbf{0}         & \mathbf{Q}^{(0,0)}
\end{bmatrix}
\in \mathbb{R}^{3N\times 12(N-1)}
\end{equation}

and

\begin{equation}\label{eq:rvec}
\mathbf{r}
\equiv
\begin{bmatrix}
\mathbf{r}^{(0)}\\
\mathbf{r}^{(1)}\\
%\mathbf{w}^{(2)}\\
\vdots\\
\mathbf{r}^{(N-1)}\\
\end{bmatrix}
\in \mathbb{R}^{3N}
\end{equation}


%%%%%%%%%%%%%%%%%%%%%%%%%%%%%%%%%%%%%%%%%%%%%%%%%%%%%%%%%%%%%%%%%%%%%%%%%%%%%
\subsubsection{Continuity conditions in internal point}
Following the Fig. \ref{fig:3DSplinePoly}, 
we can affirm that for $0 \leq n\leq N-3$,
so that
\begin{equation}\label{eq:bound1}
\mathbf{p}^{(n)}(n+1)-\mathbf{p}^{n+1}(n+1)
\approx
\mathbf{0}\in \mathbb{R}^{3},
\end{equation}

\begin{equation}\label{eq:bound2}
\left.\frac{\partial\mathbf{p}^{(n)}(t)}{\partial t}\right|_{t=n+1}
-
\left.\frac{\partial\mathbf{p}^{n+1}(t)}{\partial t}\right|_{t=n+1}
\approx\mathbf{0}\in \mathbb{R}^{3},
\end{equation}

\begin{equation}\label{eq:bound3}
\left.\frac{\partial^{2}\mathbf{p}^{(n)}(t)}{\partial t^{2}}\right|_{t=n+1}
-
\left.\frac{\partial^{2}\mathbf{p}^{n+1}(t)}{\partial t^{2}}\right|_{t=n+1}
\approx\mathbf{0}\in \mathbb{R}^{3}.
\end{equation}

%%%%%%%%%%%%%%%%%%%%%%%%%%%%%%%%%%%%%%%%%%%%%%%%%%%%%%%%%%%%%%%%%%%%%%%%%%%%%
\subsubsection{Matricial form of the continuity conditions in internal point equation's}
Using % 
the Eqs. \ref{eq:primeder0}, \ref{eq:primeder1}, \ref{eq:primeder2}, \ref{eq:primeder3a} and \ref{eq:primeder3b} in 
the Eqs. \ref{eq:bound1}, \ref{eq:bound2} and \ref{eq:bound3},
we obtain for $0 \leq n\leq N-3$
\begin{equation}
 \mathbf{A}^{(n)}(n+1) \mathbf{w}^{(n)} - \mathbf{A}^{(n+1)}(n+1) \mathbf{w}^{(n+1)} 
 \approx
 \mathbf{0} \in \mathbb{R}^{3},
\end{equation}

\begin{equation}
D_{t}\mathbf{A}^{(n)}(n+1)
\mathbf{w}^{(n)}
-
D_{t}\mathbf{A}^{(n+1)}(n+1)
\mathbf{w}^{(n+1)}
\approx\mathbf{0}\in \mathbb{R}^{3},
\end{equation}


\begin{equation}
D_{t}^{2}\mathbf{A}^{(n)}(n+1)
\mathbf{w}^{(n)}
-
D_{t}^{2}\mathbf{A}^{(n+1)}(n+1)
\mathbf{w}^{(n+1)}
=\mathbf{0}\in \mathbb{R}^{3}.
\end{equation}

Grouping in a matrix

\begin{equation}
\begin{bmatrix}
\mathbf{A}^{(n)}(n+1) & -\mathbf{A}^{(n+1)}(n+1)\\
D_{t}\mathbf{A}^{(n)}(n+1) & -D_{t}\mathbf{A}^{(n+1)}(n+1)\\
D_{t}^{2}\mathbf{A}^{(n)}(n+1) & -D_{t}^{2}\mathbf{A}^{(n+1)}(n+1)
\end{bmatrix}
\begin{bmatrix}
\mathbf{w}^{(n)}\\
\mathbf{w}^{(n+1)}
\end{bmatrix}
\approx\mathbf{0}\in \mathbb{R}^{9},
\end{equation}


using the Eqs. \ref{eq:primeder1} and \ref{eq:primeder2}, 
we know that

\begin{equation}\label{eq:Q10}
D_{t} \mathbf{A}^{(n)}(n+1)
=
\left[
\begin{array}{cccc:cccc:cccc}
0 & 1 & 2 & 3 &
0 & 0 & 0 & 0 &
0 & 0 & 0 & 0 \\
0 & 0 & 0 & 0 &
0 & 1 & 2 & 3 &
0 & 0 & 0 & 0 \\
0 & 0 & 0 & 0 &
0 & 0 & 0 & 0 &
0 & 1 & 2 & 3 
\end{array}
\right]
\equiv \mathbf{Q}^{(1,0)}\in \mathbb{R}^{3\times 12},
\end{equation}

\begin{equation}\label{eq:Q11}
D_{t} \mathbf{A}^{(n+1)}(n+1)
=
\left[
\begin{array}{cccc:cccc:cccc}
0 & 1 & 0 & 0 &
0 & 0 & 0 & 0 &
0 & 0 & 0 & 0 \\
0 & 0 & 0 & 0 &
0 & 1 & 0 & 0 &
0 & 0 & 0 & 0 \\
0 & 0 & 0 & 0 &
0 & 0 & 0 & 0 &
0 & 1 & 0 & 0 
\end{array}
\right]
\equiv \mathbf{Q}^{(1,1)}\in \mathbb{R}^{3\times 12},
\end{equation}

\begin{equation}\label{eq:Q20}
D_{t}^{2} \mathbf{A}^{(n)}(n+1)
=
\left[
\begin{array}{cccc:cccc:cccc}
0 & 0 & 2 & 6 &
0 & 0 & 0 & 0 &
0 & 0 & 0 & 0 \\
0 & 0 & 0 & 0 &
0 & 0 & 2 & 6 &
0 & 0 & 0 & 0 \\
0 & 0 & 0 & 0 &
0 & 0 & 0 & 0 &
0 & 0 & 2 & 6 
\end{array}
\right]
\equiv \mathbf{Q}^{(2,0)}\in \mathbb{R}^{3\times 12},
\end{equation}

\begin{equation}\label{eq:Q21}
D_{t}^{2} \mathbf{A}^{(n+1)}(n+1)
=
\left[
\begin{array}{cccc:cccc:cccc}
0 & 0 & 2 & 0 &
0 & 0 & 0 & 0 &
0 & 0 & 0 & 0 \\
0 & 0 & 0 & 0 &
0 & 0 & 2 & 0 &
0 & 0 & 0 & 0 \\
0 & 0 & 0 & 0 &
0 & 0 & 0 & 0 &
0 & 0 & 2 & 0 
\end{array}
\right]
\equiv \mathbf{Q}^{(2,1)}\in \mathbb{R}^{3\times 12}.
\end{equation}

Thus,
using the Eqs. \ref{eq:Q00}, \ref{eq:Q01}, \ref{eq:Q10}, \ref{eq:Q11}, \ref{eq:Q20} and \ref{eq:Q21} in 
the Eqs. \ref{eq:bound1}, \ref{eq:bound2} and \ref{eq:bound3}, 
these can be rewritten for $0 \leq n\leq N-3$

\begin{equation}
\begin{bmatrix}
\mathbf{Q}^{(0,0)} & -\mathbf{Q}^{(0,1)}\\
\mathbf{Q}^{(1,0)} & -\mathbf{Q}^{(1,1)}\\
\mathbf{Q}^{(2,0)} & -\mathbf{Q}^{(2,1)}\\
\end{bmatrix}
\begin{bmatrix}
\mathbf{w}^{(n)}\\
\mathbf{w}^{(n+1)}
\end{bmatrix}
\approx\mathbf{0}\in \mathbb{R}^{9}
\end{equation}

Finally, 
concatenating to all values for $0 \leq n\leq N-3$
in the boundary conditions in internal point equation's, we obtain

\begin{equation}\label{eq:Qmat}
\mathbf{Q}
\equiv
\begin{bmatrix}
\mathbf{Q}^{(0,0)} & -\mathbf{Q}^{(0,1)} & \mathbf{0} & \mathbf{0} & \hdots & \mathbf{0} & \mathbf{0} & \mathbf{0}\\
\mathbf{Q}^{(1,0)} & -\mathbf{Q}^{(1,1)} & \mathbf{0} & \mathbf{0} & \hdots & \mathbf{0} & \mathbf{0} & \mathbf{0}\\
\mathbf{Q}^{(2,0)} & -\mathbf{Q}^{(2,1)} & \mathbf{0} & \mathbf{0} & \hdots & \mathbf{0} & \mathbf{0} & \mathbf{0}\\ \hdashline[2pt/2pt]
\mathbf{0} & \mathbf{Q}^{(0,0)} & -\mathbf{Q}^{(0,1)} & \mathbf{0} & \hdots & \mathbf{0} & \mathbf{0} & \mathbf{0}\\
\mathbf{0} & \mathbf{Q}^{(1,0)} & -\mathbf{Q}^{(1,1)} & \mathbf{0} & \hdots & \mathbf{0} & \mathbf{0} & \mathbf{0}\\
\mathbf{0} & \mathbf{Q}^{(2,0)} & -\mathbf{Q}^{(2,1)} & \mathbf{0} & \hdots & \mathbf{0} & \mathbf{0} & \mathbf{0}\\ \hdashline[2pt/2pt]
\vdots     & \vdots             & \vdots             & \vdots     & \vdots & \vdots     & \vdots     & \vdots    \\ \hdashline[2pt/2pt]
\mathbf{0} & \mathbf{0}         & \mathbf{0}         & \mathbf{0} & \hdots & \mathbf{0} & \mathbf{Q}^{(0,0)} & -\mathbf{Q}^{(0,1)}\\
\mathbf{0} & \mathbf{0}         & \mathbf{0}         & \mathbf{0} & \hdots & \mathbf{0} & \mathbf{Q}^{(1,0)} & -\mathbf{Q}^{(1,1)}\\
\mathbf{0} & \mathbf{0}         & \mathbf{0}         & \mathbf{0} & \hdots & \mathbf{0} & \mathbf{Q}^{(2,0)} & -\mathbf{Q}^{(2,1)}\\
\end{bmatrix}
\in \mathbb{R}^{9(N-2)\times 12(N-1)}
\end{equation}

\begin{equation}
\mathbf{Q}
\mathbf{w}
\approx\mathbf{0}\in \mathbb{R}^{9(N-2)}
\end{equation}


%%%%%%%%%%%%%%%%%%%%%%%%%%%%%%%%%%%%%%%%%%%%%%%%%%%%%%%%%%%%%%%%%%%%%%%%%%%%%
%%%%%%%%%%%%%%%%%%%%%%%%%%%%%%%%%%%%%%%%%%%%%%%%%%%%%%%%%%%%%%%%%%%%%%%%%%%%%

\subsection{Square curvature of a 3D parametric cubic spline}
\label{sec:curvaturemain}
We know by the Eq. (\ref{eq:curvaturekn}) that

\begin{equation}
\mathcal{K}_{(n)}(t)
=
\frac{\left\|{\mathbf{p}^{(n)}}'(t) \times {\mathbf{p}^{(n)}}''(t) \right\|}
{\left\|{\mathbf{p}^{(n)}}'(t)\right\|^{3}}
\end{equation}

If we calculate the square curvature $\mathcal{K}_{(n)}^{2}(t)$ for the boundary cases of the curve $\mathbf{p}^{(n)}(t)$, 
when $t\equiv n$ and $t\equiv n+1$,
then we obtain $\mathcal{K}_{(n)}^{2}(n)$ and $\mathcal{K}_{(n)}^{2}(n+1)$, respectively
\begin{equation}\label{eq:curvature:K2nn}
\mathcal{K}_{(n)}^{2}(n)
=
\frac{\left\|
\left( \mathbf{Q}^{(1,1)} \mathbf{w}^{(n)} \right)
\times 
\left( \mathbf{Q}^{(2,1)} \mathbf{w}^{(n)} \right)
\right\|^{2}}
{\left\| \mathbf{Q}^{(1,1)} \mathbf{w}^{(n)} \right\|^{6}}
\end{equation}

\begin{equation}\label{eq:curvature:K2nn1}
\mathcal{K}_{(n)}^{2}(n+1)
=
\frac{\left\|
\left( \mathbf{Q}^{(1,0)} \mathbf{w}^{(n)} \right)
\times 
\left( \mathbf{Q}^{(2,0)} \mathbf{w}^{(n)} \right)
\right\|^{2}}
{\left\| \mathbf{Q}^{(1,0)} \mathbf{w}^{(n)} \right\|^{6}}
\end{equation}


If we use the Eq. (\ref{eq:cuvaturepartial}) to differentiate the Eqs. (\ref{eq:curvature:K2nn}) and (\ref{eq:curvature:K2nn1})
with respect to $\mathbf{w}^{(n)}$, 
we obtain the Eqs. (\ref{eq:curvature:der:K2nn}) and (\ref{eq:curvature:der:K2nn1}), respectively.

\begin{equation}\label{eq:curvature:der:K2nn}
\begin{split}
\frac{
\partial 
\mathcal{K}_{(n)}^{2}(n)
}
{
\partial \mathbf{w}^{(n)}
}
& = 
2
\frac{
\left\|{\mathbf{p}^{(n)}}''(n)\right\|^2
\mathbf{Q}^{(1,1)T} {\mathbf{p}^{(n)}}'(n)
+
\left\|{\mathbf{p}^{(n)}}'(n)\right\|^2
\mathbf{Q}^{(2,1)T} {\mathbf{p}^{(n)}}''(n)
}
{\left\| {\mathbf{p}^{(n)}}'(n) \right\|^{6}}\\[10pt]
& - 
2
\frac
{
\left(
{{\mathbf{p}^{(n)}}'(n)}^{T}
{\mathbf{p}^{(n)}}''(n)
\right)
\left(
\mathbf{Q}^{(1,1)T}{\mathbf{p}^{(n)}}''(n)
+
\mathbf{Q}^{(2,1)T}{\mathbf{p}^{(n)}}'(n)
\right)
}
{\left\| {\mathbf{p}^{(n)}}'(n) \right\|^{6}}\\[10pt]
& - 
6
\frac
{
\mathcal{K}_{(n)}^{2}(n)
\mathbf{Q}^{(1,1)T}{\mathbf{p}^{(n)}}'(n)
}
{\left\| {\mathbf{p}^{(n)}}'(n) \right\|^{2}}
\end{split}
\end{equation}

and

\begin{equation}\label{eq:curvature:der:K2nn1}
\begin{split}
\frac{
\partial 
\mathcal{K}_{(n)}^{2}(n+1)
}
{
\partial \mathbf{w}^{(n)}
}
& = 
2
\frac{
\left\|{\mathbf{p}^{(n)}}''(n+1)\right\|^2
\mathbf{Q}^{(1,0)T} {\mathbf{p}^{(n)}}'(n+1)
+
\left\|{\mathbf{p}^{(n)}}'(n+1)\right\|^2
\mathbf{Q}^{(2,0)T} {\mathbf{p}^{(n)}}''(n+1)
}
{\left\| {\mathbf{p}^{(n)}}'(n+1) \right\|^{6}}\\[10pt]
& - 
2
\frac
{
\left(
{{\mathbf{p}^{(n)}}'(n+1)}^{T}
{\mathbf{p}^{(n)}}''(n+1)
\right)
\left(
\mathbf{Q}^{(1,0)T}{\mathbf{p}^{(n)}}''(n+1)
+
\mathbf{Q}^{(2,0)T}{\mathbf{p}^{(n)}}'(n+1)
\right)
}
{\left\| {\mathbf{p}^{(n)}}'(n+1) \right\|^{6}}\\[10pt]
& - 
6
\frac
{
\mathcal{K}_{(n)}^{2}(n+1)
\mathbf{Q}^{(1,0)T}{\mathbf{p}^{(n)}}'(n+1)
}
{\left\| {\mathbf{p}^{(n)}}'(n+1) \right\|^{2}}
\end{split}
\end{equation}

\begin{comment}
Finally

\begin{equation}
\frac{
\partial 
\mathcal{K}_{(n)}^{2}(n+1)
}
{
\partial \mathbf{w}^{(n)}
}
\end{equation}
\end{comment}

%%%%%%%%%%%%%%%%%%%%%%%%%%%%%%%%%%%%%%%%%%%%%%%%%%%%%%%%%%%%%%%%%%%%%%%%%%%%%
%%%%%%%%%%%%%%%%%%%%%%%%%%%%%%%%%%%%%%%%%%%%%%%%%%%%%%%%%%%%%%%%%%%%%%%%%%%%%
\subsection{Cost function in 3D parametric cubic spline}

%%%%%%%%%%%%%%%%%%%%%%%%%%%%%%%%%%%%%%%%%%%%%%%%%%%%%%%%%%%%%%%%%%%%%%%%%%%%%
\subsubsection{Cost function $E_{1}(\mathbf{w})$ of fitting the cubic spline in the points $\mathbf{r}^{(n)}$}

Following the explanation in the Section \ref{sec:boundarycubic},
the equation that should be fulfilled to fit the cubic spline in the points can be represented in the next equation

\begin{equation}
\begin{bmatrix}
\mathbf{P}\\
\mathbf{Q}
\end{bmatrix}
\mathbf{w}
\approx
\begin{bmatrix}
\mathbf{r}\\
\mathbf{0}
\end{bmatrix}
\in \mathbb{R}^{12(N-1)-6}
\end{equation}

Defining the cost function $E_{1}(\mathbf{w})$ of fitting of cubic spline in the points $\mathbf{r}^{(n)}$,
$\forall 0\leq n\leq N-1$

\begin{equation}\label{eq:costfunc1}
E_{1}(\mathbf{w})
=
\left\|
\begin{bmatrix}
\mathbf{P}\\
\mathbf{Q}
\end{bmatrix}
\mathbf{w}
-
\begin{bmatrix}
\mathbf{r}\\
\mathbf{0}
\end{bmatrix}
\right\|_{\mathbf{D}}^{2}
\end{equation}
where $\mathbf{D}\in \mathbb{R}^{(12N-18)\times(12N-18)}$ is a diagonal matrix with the weight of elements of Eq. (\ref{eq:costfunc1}), 
$\left\|\mathbf{a}\right\|_{\mathbf{D}}^{2}\equiv \mathbf{a}^{T}\mathbf{D}\mathbf{a}$.

\begin{equation}
\mathbf{D} 
= 
diag(
\begin{bmatrix}
\mathbf{d}_{\mathbf{r}}\\
\mathbf{d}_{\mathbf{c}}
\end{bmatrix}
)
\end{equation}
being
\begin{itemize}
\item Weight of boundary conditions $\mathbf{d}_{\mathbf{r}} \in \mathbb{R}^{3N}$, 
as demonstrated in example: $\mathbf{d}_{\mathbf{r}} \equiv \gamma_{\mathbf{r}} \mathbf{1}_{3N}$, where $\gamma_{\mathbf{r}}$ is a scalar selected by the researcher.
\item Weight of continuity conditions $\mathbf{d}_{\mathbf{c}} \in \mathbb{R}^{9(N-2)}$,
as demonstrated in example: 
$\mathbf{d}_{\mathbf{c}} \equiv \begin{bmatrix}\gamma_{\mathbf{c0}} \mathbf{1}_{3}^{T} & \gamma_{\mathbf{c1}} \mathbf{1}_{3}^{T} & \gamma_{\mathbf{c2}} \mathbf{1}_{3}^{T} \end{bmatrix}^{T} * (N-2)$, 
where $\gamma_{\mathbf{c0}}$, $\gamma_{\mathbf{c1}}$ and $\gamma_{\mathbf{c2}}$ are scalars selected by the researcher 
to weighted the continuity conditions of order 0, 1 and 2.
\end{itemize}


Applying the derivative in relation to vector $\mathbf{w}$
\cite[pp. 11]{petersen2008matrix}
in the cost function $E_{1}(\mathbf{w})$ of Eq. \ref{eq:costfunc1}, 
we obtain

\begin{equation}\label{eq:DE1}
\frac{\partial E_{1}(\mathbf{w})}{\partial \mathbf{w}}
=
2
\begin{bmatrix}
\mathbf{P}\\
\mathbf{Q}
\end{bmatrix}^{T}
\mathbf{D}
\left(
\begin{bmatrix}
\mathbf{P}\\
\mathbf{Q}
\end{bmatrix}
\mathbf{w}
-
\begin{bmatrix}
\mathbf{r}\\
\mathbf{0}
\end{bmatrix}
\right)
\end{equation}

%%%%%%%%%%%%%%%%%%%%%%%%%%%%%%%%%%%%%%%%%%%%%%%%%%%%%%%%%%%%%%%%%%%%%%%%%%%%%
\subsubsection{Cost function $E_{2}(\mathbf{w})$ of curvature the cubic spline in the points $\mathbf{r}^{(n)}$}

Following the explanation in the Section \ref{sec:curvaturemain},
the equation relative to the curvature that should be minimized can be represented in the next equation




\begin{equation}
E_{2}(\mathbf{w})
=
\frac{1}{2(N-1)}
\sum\limits_{n=0}^{N-2}
\left\{
\mathcal{K}_{(n)}^{2}(n)
+
\mathcal{K}_{(n)}^{2}(n+1)
\right\}.
\end{equation}

Derivating in function of $\mathbf{w}$

\begin{equation}
\frac{\partial E_{2}(\mathbf{w})}{\partial \mathbf{w}^{(n)}}
=
\frac{1}{2(N-1)}
\left\{
\frac{
\partial 
\mathcal{K}_{(n)}^{2}(n)
}{\partial \mathbf{w}^{(n)}}
+
\frac{
\partial 
\mathcal{K}_{(n)}^{2}(n+1)
}{\partial \mathbf{w}^{(n)}}
\right\}
\end{equation}

\begin{equation}\label{eq:DE2}
\frac{\partial E_{2}(\mathbf{w})}{\partial \mathbf{w}}
=
\begin{bmatrix}
\frac{\partial E_{2}(\mathbf{w})}{\partial \mathbf{w}^{(0)}}\\[4pt]
\frac{\partial E_{2}(\mathbf{w})}{\partial \mathbf{w}^{(1)}}\\[4pt]
\vdots\\[4pt]
\frac{\partial E_{2}(\mathbf{w})}{\partial \mathbf{w}^{(N-2)}}
\end{bmatrix}
=
\frac{1}{2(N-1)}
\begin{bmatrix}
%
\frac{
\partial 
\mathcal{K}_{(0)}^{2}(0)
}{\partial \mathbf{w}^{(0)}}
+
\frac{
\partial 
\mathcal{K}_{(0)}^{2}(1)
}{\partial \mathbf{w}^{(0)}}\\[4pt]
%
\frac{
\partial 
\mathcal{K}_{(1)}^{2}(1)
}{\partial \mathbf{w}^{(1)}}
+
\frac{
\partial 
\mathcal{K}_{(1)}^{2}(2)
}{\partial \mathbf{w}^{(1)}}\\[4pt]
%
\vdots\\[4pt]
\frac{
\partial 
\mathcal{K}_{(N-2)}^{2}(N-2)
}{\partial \mathbf{w}^{(N-2)}}
+
\frac{
\partial 
\mathcal{K}_{(N-2)}^{2}(N-1)
}{\partial \mathbf{w}^{(N-2)}}\\
%
\end{bmatrix}
.
\end{equation}

%%%%%%%%%%%%%%%%%%%%%%%%%%%%%%%%%%%%%%%%%%%%%%%%%%%%%%%%%%%%%%%%%%%%%%%%%%%%%
\subsubsection{Total cost function of 3D parametric cubic spline}
\label{sec:TotalCostFunc}

Defining the total cost function $E(\mathbf{w})$ as

\begin{equation}
E(\mathbf{w})=E_{1}(\mathbf{w})+\beta E_{2}(\mathbf{w})
\end{equation}

\begin{equation}
E(\mathbf{w})
=
\left\|
\begin{bmatrix}
\mathbf{P}\\
\mathbf{Q}
\end{bmatrix}
\mathbf{w}
-
\begin{bmatrix}
\mathbf{r}\\
\mathbf{0}
\end{bmatrix}
\right\|_{\mathbf{D}}^{2}
+\beta 
\left[
\frac{1}{2(N-1)}
\sum\limits_{n=0}^{N-2}
\left\{
\mathcal{K}_{(n)}^{2}(n)
+
\mathcal{K}_{(n)}^{2}(n+1)
\right\}
\right].
\end{equation}

where 
\begin{itemize}
\item the constant matrix $\mathbf{P} \in \mathbb{R}^{3N\times 12(N-1)}$ is defined in Eq. (\ref{eq:Pmat}),
\item the constant matrix $\mathbf{Q} \in \mathbb{R}^{9(N-2)\times 12(N-1)}$ is defined in Eq. (\ref{eq:Qmat}),
\item the diagonal matrix $\mathbf{D} \in \mathbb{R}^{(12N-18)\times (12N-18)}$ contains hyperparameters that assign weights to boundary and continuity conditions,
\item the constant vector $\mathbf{r} \in \mathbb{R}^{3N}$ is an input position data described in Eq. (\ref{eq:rvec}),
\item the vector $\mathbf{w} \in \mathbb{R}^{12(N-1)}$ has the parameters of spline as described in Eq. (\ref{eq:wvector}),
\item the scalar $\beta$ serves as a hyperparameter to balance the significance of $E_{2}(\mathbf{w})$ relative to $E_{1}(\mathbf{w})$,
\item the scalar $\mathcal{K}_{(n)}^{2}(n)$ is determined using Eq. (\ref{eq:curvature:K2nn}) and 
\item the scalar $\mathcal{K}_{(n)}^{2}(n+1)$ is determined using Eq. (\ref{eq:curvature:K2nn1}).
\end{itemize}
%%%%%%%%%%%%%%%%%%%%%%%%%%%%%%%%%%%%%%%%%%%%%%%%%%%%%%%%%%%%%%%%%%%%%%%%%%%%%
\subsection{Parameter calculus of 3D parametric cubic spline}
\label{sec:solvecubicspline}
Using the gradient descent technique, we obtain

\begin{equation}
\mathbf{w}_{i+1}
\leftarrow 
\mathbf{w}_{i}
-
\alpha
\left.
\frac{\partial 
\left\{
E_{1}(\mathbf{w})+\beta E_{2}(\mathbf{w})
\right\}
}{\partial \mathbf{w}}
\right|_{\mathbf{w}=\mathbf{w}_{i}}.
\end{equation}

Applying the Eqs. \ref{eq:DE1}, \ref{eq:DE2} we obtain

\begin{equation}
\mathbf{w}_{i+1}
\leftarrow 
\mathbf{w}_{i}
-
\alpha
\left\{
\left.
\frac{\partial E_{1}(\mathbf{w})}{\partial \mathbf{w}}
\right|_{\mathbf{w}=\mathbf{w}_{i}}
+
\beta
\left.
\frac{\partial E_{2}(\mathbf{w})}{\partial \mathbf{w}}
\right|_{\mathbf{w}=\mathbf{w}_{i}}
\right\}
\end{equation}

\begin{equation}
\mathbf{w}_{i+1}
\leftarrow 
\mathbf{w}_{i}
-
\alpha
\left\{
2
\begin{bmatrix}
\mathbf{P}\\
\mathbf{Q}
\end{bmatrix}^{T}
\mathbf{D}
\left(
\begin{bmatrix}
\mathbf{P}\\
\mathbf{Q}
\end{bmatrix}
\mathbf{w}_{i}
-
\begin{bmatrix}
\mathbf{r}\\
\mathbf{0}
\end{bmatrix}
\right)
+
\frac{\beta}{2(N-1)}
\begin{bmatrix}
%
\frac{
\partial 
\mathcal{K}_{(0)}^{2}(0)
}{\partial \mathbf{w}^{(0)}}
+
\frac{
\partial 
\mathcal{K}_{(0)}^{2}(1)
}{\partial \mathbf{w}^{(0)}}\\[4pt]
%
\frac{
\partial 
\mathcal{K}_{(1)}^{2}(1)
}{\partial \mathbf{w}^{(1)}}
+
\frac{
\partial 
\mathcal{K}_{(1)}^{2}(2)
}{\partial \mathbf{w}^{(1)}}\\[4pt]
%
\vdots\\[4pt]
\frac{
\partial 
\mathcal{K}_{(N-2)}^{2}(N-2)
}{\partial \mathbf{w}^{(N-2)}}
+
\frac{
\partial 
\mathcal{K}_{(N-2)}^{2}(N-1)
}{\partial \mathbf{w}^{(N-2)}}\\
%
\end{bmatrix}
\right\}
\end{equation}

where $\alpha$ and $\beta$ are learning hyper-parameters and 
$\mathbf{w}_{0}$ can be a random vector or can
be calculated following a 3D parametric linear spline,
see Section \ref{sec:solvelinearspline}.
The iteration follows until a maximum number of iterations $i$ or until it reaches a
minimum defined error $E_1(\mathbf{w}_{i})$. 

