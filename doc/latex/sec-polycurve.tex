\section{Piecewise polynomial curve $\mathbf{\bar{p}}(t)$}
\label{sec:curvePiecewise}
A parametrized curve $\mathbf{\bar{p}}(t)$ in real space can be formally defined as:

\begin{equation}
\mathbf{\bar{p}}: D \subseteq \mathbb{R} \to \mathbb{R}^M
\end{equation}

where: 

\begin{itemize}
\item $D$ is the domain of the parameterization, i.e., the set of values of $t$ for which $\mathbf{\bar{p}}(t)$ is defined.  
\item $\mathbb{R}^M$ represents the space where the curve is contained.
Being $M$ the dimension of space.
\end{itemize}

If the curve is defined only for a finite interval $t \in [a, b]$, then the domain is $D = [a, b]$. Otherwise, if the curve is defined for all real numbers, then $D = \mathbb{R}$.  




Each segment of the curve, defined by $N$ points, is given by the function $\mathbf{p}^{(n)}(t)$, where $0 \leq n < N-1$.  
Thus, the complete curve is defined as a piecewise function:

\begin{equation}\label{eq:phatt}
\mathbf{\bar{p}}(t) =
\begin{cases} 
\mathbf{p}^{(0)}(t), & t_0 \leq t < t_1 \\
\mathbf{p}^{(1)}(t), & t_1 \leq t < t_2 \\
\vdots \\
\mathbf{p}^{(N-2)}(t), & t_{N-2} \leq t \leq t_{N-1}
\end{cases}
\end{equation}

where
\begin{itemize}
\item $t_0, t_1, \dots, t_N $ define the parameter intervals for each segment.  
\item Each function $\mathbf{p}^{(n)}: [t_{n},t_{n+1}] \subseteq \mathbb{R} \to \mathbb{R}^M$ represents a segment of the curve and can be a polynomial, spline, or any other interpolating function.  
\end{itemize}


%%%%%%%%%%%%%%%%%%%%%%%%%%%%%%%%%%%%%%%%%%%%%%%%%%%%%%%%%%%%%%%%%%%%%%%%%%%%%
%%%%%%%%%%%%%%%%%%%%%%%%%%%%%%%%%%%%%%%%%%%%%%%%%%%%%%%%%%%%%%%%%%%%%%%%%%%%%
\subsection{Continuity conditions}\label{sec:continuity}
Continuity conditions must be satisfied to ensure a smooth transition between segments. 
Given a piecewise-defined curve $\mathbf{\bar{p}}(t)$ as Eq.(\ref{eq:phatt}), 
each segment $\mathbf{p}^{(n)}(t)$ must satisfy specific continuity conditions at the junctions $t_n$ to ensure smoothness:
\begin{itemize}
\item \textbf{Position continuity (\( C^0 \) continuity):}  
   \[
   \mathbf{p}^{(n)}(t_{n+1}) = \mathbf{p}^{(n+1)}(t_{n+1}), \quad \forall n \in \{0, \dots, N-2\}.
   \]
   This ensures that the segments connect without gaps.

\item \textbf{First derivative continuity (\( C^1 \) continuity):}  
   \[
   \mathbf{p}^{(n)'}(t_{n+1}) = \mathbf{p}^{(n+1)'}(t_{n+1}), \quad \forall n \in \{0, \dots, N-2\}.
   \]
   This ensures a smooth velocity transition, preventing sharp corners.

\item \textbf{Second derivative continuity (\( C^2 \) continuity):}  
   \[
   \mathbf{p}^{(n)''}(t_{n+1}) = \mathbf{p}^{(n+1)''}(t_{n+1}), \quad \forall n \in \{0, \dots, N-2\}.
   \]
   This ensures a smooth acceleration transition, avoiding abrupt changes in curvature.
\item \textbf{End-point interpolation:}  
   Each segment \(\mathbf{p}^{(n)}(t)\) must pass through a specified point \(\mathbf{r}^{(n)}\) at its endpoints:
   \[
   \mathbf{p}^{(n)}(t_n) = \mathbf{r}^{(n)}, \quad \mathbf{p}^{(n)}(t_{n+1}) = \mathbf{r}^{(n+1)}, \quad \forall n \in \{0, \dots, N-2\}.
   \]
   This ensures that each segment reaches the prescribed points in space.
\end{itemize}

\begin{comment}
For a \textbf{natural cubic spline}, additional boundary conditions are often imposed, such as:
\begin{itemize}
\item \textbf{Clamped spline}: The first derivative at the endpoints is specified
  \[
  \mathbf{p}^{(0)'}(t_0) = \mathbf{v}_0, \quad \mathbf{p}^{(N-1)'}(t_N) = \mathbf{v}_N.
  \]
\item \textbf{Natural spline}: The second derivative at the endpoints is set to zero
  \[
  \mathbf{p}^{(0)''}(t_0) = 0, \quad \mathbf{p}^{(N-1)''}(t_N) = 0.
  \]
\end{itemize}
\end{comment}
These conditions ensure the smoothness and stability of the cubic spline in 3D space.


%%%%%%%%%%%%%%%%%%%%%%%%%%%%%%%%%%%%%%%%%%%%%%%%%%%%%%%%%%%%%%%%%%%%%%%%%%%%%
%%%%%%%%%%%%%%%%%%%%%%%%%%%%%%%%%%%%%%%%%%%%%%%%%%%%%%%%%%%%%%%%%%%%%%%%%%%%%
\subsection{Curvature}
\label{sec:curvaturepn}
The curvature $\mathcal{K}_{(n)}(t)$ of a 3D parametric function, 
$\mathbf{p}^{(n)}(t)\in \mathbb{R}^{3}$ with parameter $t$, 
can be calculated 
\cite[pp. 21]{toponogov2006differential} 
using the following equation,

\begin{equation}\label{eq:curvaturekn}
\mathcal{K}_{(n)}(t)
=
\frac{\left\|\frac{\partial \mathbf{p}^{(n)}(t)}{\partial t} \times \frac{\partial^2 \mathbf{p}^{(n)}(t)}{\partial t^2} \right\|}
{\left\|\frac{\partial \mathbf{p}^{(n)}(t)}{\partial t}\right\|^{3}}
\equiv
\frac{\left\|{\mathbf{p}^{(n)}}'(t) \times {\mathbf{p}^{(n)}}''(t) \right\|}
{\left\|{\mathbf{p}^{(n)}}'(t)\right\|^{3}}
\end{equation}